\documentclass[]{article}

%opening
\title{Can computer think?}
\author{Girish Modiletappa}

\begin{document}

\maketitle

\begin{abstract}
Thinking is a non-trivial process to make a computer develop one. In its true essence 'thinking' is what a being is capable of. A sophisticated process which involves past knowledge
and free will.
\end{abstract}

\section{Any form of computation is mechanical and doesn't need thinking}
\subsection{If computers can think then what qualities would they express or possess?}
\paragraph{}They would converge to a decision based on experiences and presuppositions. There would be a bias
in its choices for rationality may not always be an outcome.
\subsection{What is thinking, where does it originate and how does it perpetuate?}
\paragraph{}
Thought as a random occurrence in our mind would have a more simpler circuitry for computers. Can they
rewire their thoughts to decide and make choices through their own free will? This leads to self awareness
and most importantly consciousness.
\subsection{What is consciousness and how are thoughts related to it?} 
\paragraph{}
When we say that a computer can think and take actions we say that it does with minimal programming or human
intervention. It can sense the surroundings through its sensors. Interpret the physical quantities and formulate
actions based on experience.
\subsection{What makes a 'layman' machine into a 'thoughtful' machine?}
\paragraph{}
An example that could be thought here is of a lawn mower. This cuts down grass to keep the garden tidy. So, a 
lawn mower operated by a gardener would just cut down whatever it can get in its way. Insects, grass, weed, flowers...
anything and everything. The onus is on the gardener to control its movement and how much time it spends at a particular
location. A further advancement to this machine would be that it could operate in a certain area of action without gardener's
involvement. The gardener only fixes the boundaries for mower. A geometric 2D area for the mower to function.
The mower is now capable of measuring distance from its starting point and gauge the area of operation. Also, the gardener may
fix operational hours and just let it be. This mower now operates under certain program but still doesn't make decisions on which
grass to cut, whether to skip any earthworms and mow parts depending on a optimal path. The mower with an autonomous computer
that could think would have to think like a gardener. The gardener senses environment around him and ventures mowing based on external
factors, history and soil conditions. The machine would have to rewire and register memories of new events and pictures of the
vegetation around it. It needs to think and compute surrounding to not get in contact with any insect or human. What happens when unknown
conditions in environment affects the machine? Would it know when to stop and restart? Can it make a decision on its own? Can machine
understand and allocate for its own survival and longevity? Can a machine in itself consider it as a living being and take action to 
nurture itself and avoid conflict with nature for its survival? The neural network of a machine and its dynamic wiring to make a choice
or action depends on this process of thinking known as a \textit{thought-process}. It is the key to any problem solving using its 
experience paves way to newer solutions. Only by past solution to similar problems can humans tend to innovate on solving existing ones.
Moreover, after finding a solution how can machine evaluate and reassess for its correctness without a verification model designed autonomously.
\section{Turing test and its effectiveness}
\subsection{Is the Turing test meaningful and valid?}
\paragraph{}
Turing suggested that if a computer and a human being were hidden behind a screen, and another
human being were given the task of interrogating each of them, it would be reasonable to conclude that the computer was conscious if the interrogator could not distinguish it
from the human being. There have been many variations of the Turing test proposed, some by Turing himself, and there are annual contests based on Turing test. Thus far, no computer has
passed the Turing test (by general consensus), although some have come close. Several plausible characteristics have been proposed — free will, restricted access (only the thinker experiences his thoughts), incorrigibility (only the thinker knows with certainty the content of his thought), qualia (raw sensory experience), etc
\end{document}
